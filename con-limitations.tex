\section{Limitations of the Study}
\label{sec:con-limitations}
There were several limitations of significance throughout this thesis. The limited size of the dataset and the chosen emotions, as the acoustic features, may have impacted the results. A bigger dataset with more emotions and acoustic features could have given important insight and clearer results if included.

Another important limitation to mention is the spontaneous nature of the speech gathered from the interviews. Using an acted dataset may have resulted in emotions being more accurately and stronger identified, whereas the interview setting may have led to some emotions being left undetected as they may have been expressed too subtly. This may be due to conversational speech being more subtle than acted speech, reducing levels of clear emotions and vocal markers.
Worth noting is also that the Hume AI emotion scores used for comparisons, should not be considered ground truth, but more as a something to compare against.  

This study also used self-reported emotion scores as a comparison, which may have given a large variety across the dataset due to the participants own self-perception.
