
\section{Result Discussion RQ2}
For the second research question in this thesis, the aim was to investigate the similarities and differences between the two AI models Hume AI, a speech-based model, and NLP Cloud, a text-based model. This was measured when labelling five different emotions in semi-structured interviews.

\label{subsec:RQ2interpretation}
In the comparison between the two models, the system overall seemed to show some levels of agreement for certain emotions. Using both descriptive statistics and visual analyses to calculate the differences, an overall comparison showed that the mean emotion scores differ across the two models.
In figure~\ref{fig:rq2_sent_grouped_bar}, the results of the entire dataset divided between the positive and negative recordings were presented and the results highlighted some key differences in the interpretation of the emotional content between NLP Cloud and Hume AI. NLP Cloud appeared to better capture the contextual nuances, as scores for the positive clips had higher scores for joy and surprise, while negative recordings had a higher score for anger and sadness. 
Hume on the other hand, showed potentially misleading results. A significantly higher emotion score for anger and a somewhat higher score for sadness and fear in the positive recordings, while joy appeared significantly higher score for the negative recordings. This may be due to some signals being misinterpreted when speaking of both positive and negative topics. For example, some participants may have talked about the negative topics in an ironic tone or with sarcasm or expressed nervous laughter, which would be hard for a speech-based model to differentiate and pick up on without the textual context.
Another possible explanation may be due to pitch variations. Earlier research found that prosodic features like pitch are informative for arousal detection \autocite{Soleymani2017}. Pitch is one of the key features for emotion recognition in audio and it is possible participants of the interviews may have spoken with a high pitch even if not using a very descriptive language. Hume AI could have interpreted a high pitch as emotional intensity, possibly explaining the high score of joy in the negative recordings and the high score of anger in the positive recordings.

Joy being highly scored by NLP Cloud in the positive recordings indicates that joy may not have been as easily identified in speech-based emotion analysis, as the textual context may have conveyed a more positive tone from the text than what appeared in the voice. In contrast, anger and fear appeared to have been more effectively captured by the speech-based emotion detection in the positive clips, possibly suggesting that someone might sound angry or fearful even though they may not be experiencing these emotions in the moment. This is further backed up by earlier research stating specific words like “amazing” holds more intensity for emotions than other words like “leaves” \autocite{Chauhan2024}. The participants in the interviews may have used very descriptive language when describing positive experiences, which may be a reason for the high score for joy by NLP Cloud.

Based on the Pearson correlation analysis in table~\ref{tab:corr_all}, showing the association between the text-based and speech-based emotion recognition models for all recordings, the strongest alignments were shown for joy and anger. As no further strong correlations or statistically significant p-values were found in the other emotions, several factors may account for this result. For example, joy and anger are distinct emotions, while sadness, fear and surprise may likely involve more subtle cues and contextual factors. These emotions being more complex and seemingly more difficult to detect, may have contributed to the lower consistency across the two models for these specific emotions. Research have found that acted speech involves a stronger degree of intensity than spontaneous speech, which makes it more difficult to recognize emotions in this format \autocite{Chakraborty2016}. With spontaneous speech from interviews was used in this study, it’s very possible the models failed to detect the low-intensity emotions.

For the positive recordings only, there were two emotion showing significant correlations between the two models, joy and sadness, whereas for the negative recordings the only emotion that showed a significant correlation was joy. These results suggests that joy is the easiest emotion to detect for the models regardless of the context, and that anger, fear and surprise may be too complex to detect, although there are several possible reasons as to why they have low or inconsistent correlations. The way that the interviews are set up and the fact that the participants of the interviews talk about situations and feelings they have lived through in the past, may have resulted in emotions being expressed in a subdued way. Talking about a time where you felt fear or surprise might not be translated as strongly when time has passed, as it would in the moment when the emotions were felt. It is possible the interviewees did not feel or express strong emotions when speaking about different situations, and as neutral emotions have been found harder to detect according to earlier research \autocite{Cao2015}, it is possible some emotions have remained undetected or incorrectly detected. With this in mind, the lack of correlations for more complex emotions suggest that the interview format may have been insufficient to draw out more nuanced emotional responses. Alternatively, the AI models used may have some limitations in detecting subtle emotions.

For the full dataset, paired t-tests showed no significant differences for the mean score of the emotions across the dataset for all emotions except fear. Although the t-test indicated that while the systems do not align on detecting patterns for fear, Hume AI consistently rates the fear higher than NLP Cloud. Possible explanations for this result may reflect the differences in how emotions are conveyed and detected in the different models, whereas Hume AI possibly could have captured the more subtle vocal indicators that might not have been as easily expressed or detected in text.

T-test for the positively oriented interviews in comparison to the negative oriented interviews revealed notable findings. For the positive interviews, significant differences between the models were found for all emotions with the exception of surprise, while NLP on the other hand overestimated joy significantly. This may be explained by the complexity of emotions and emotional expression. In the positive interviews, the participants discussed joyful topics, and while this may have been detected for the text-based emotion recognition, the vocal tone could reveal more subtle cues in the tone, rhythm and pitch. For the positive interviews, the participants may have had a lower and more neutral tone and pitch than an actor acting out happiness, which could be one explanation for this result, which also can be explained by earlier research stating neutrality makes detection of emotions more difficult \autocite{Cao2015}.

For the negative interviews, significant differences were only identified for joy and sadness, where Hume rated joy with a higher score, and NLP rated sadness higher. This indicates a better alignment for the different models for the analyses made for the negatively oriented interviews. Possible reasons being people participating in the interviews may have used overly positive language out of politeness, even if the content of the words may have been negative, which would explain why Hume AI detected joy from negative oriented interviews.
