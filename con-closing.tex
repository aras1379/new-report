\section{Final Conclusion}
\label{sec:con-final-conclusion}
This thesis investigated AI based emotion recognition in textual and vocal content in the Swedish language. The results showed some similarities with existing research on vocal markers in Swedish language and strongly demonstrated the importance of using more than one modality to improve the accuracy of the emotion recognition, especially for more complex emotions. 

All of the findings gathered from the research in this thesis help contribute to the growing fields of affective computing and natural language processing. With technology becoming an increasingly larger part of society, understanding human emotions is a step towards AI becoming more effective and empathetic, which can help mold the future of education, health care and human-computer interactions.
