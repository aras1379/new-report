\section{Method Discussion}
KAN SKRIVA ALLT SOM ETT BEHÖVS INGA RUBRIKER HÄR TROR JAG 

\medskip
\medskip

\subsection{RQ1 Methodological Considerations}
To answer RQ1, the methodological approach involved analysis of emotional expression for
vocal markers in Swedish speech in comparison to AI based emotion recognition models. The
idea was to analyse emotions in a clip in its entirety and find correlations, which had some
differing results, but it proved to be a notable strength to execute the analysis on a segment
level to capture emotional fluctuations in a more dynamic way. While this offers another
another perspective, this approach introduced challenges of its own in having some inconsistent
emotions not aligning completely across the segments. Output from Hume is pre-segmented with frequency variation of 1-4 seconds, single clip dependent. Vocal features were segmented into 2.5 second timeframes used for all clips, leading to divergencies in segment length. For more accurate results, it should have been adjusted separately for each clip before segmented analysis. The methodological approach for the RQ1 was partially fulfilled by identifying some vocal fluctuations in emotions, while also revealing challenges in both segment-level and average values. Further insights could have been provided with utilizing the rule-based functionality for emotion grouping segment wise as well, to find if grouped features could provide higher correlation results for Hume labelling and acoustic markers. 

Comparing the results with the prior study on Swedish vocal markers indicate some similarities and patterns providing valuable information to this study, even if several correlations and patterns remain weak. Mainly focusing on this study as reference may create bias, by the reason that there is vastly limited prior research on Swedish in this field. Additionally, this study employed a larger number of vocal features than extracted for these results. The three frequency formants were included in both categorisation functions yet excluded in the overall data analysis. Beyond these, the Swedish study included 14 additional features, some of these were not possible to extract with Praat Parselmouth and therefore excluded. Voiced- and unvoiced length were not relevant for this study since the recordings were edited, including deleting some silent moments. Not including the full set of features is a clear limitation for the results. [ADD INSIGHTS ABOUT THESE VOCAL MARKERS FROM STUDY HERE AND WHAT THEY COULD HAVE PROVIDED]. The data collection used in this study consists of interviews capturing spontaneous speech, contrasting the referred Swedish study based on pre-defined sentences, repeated by four actors. With 15 participants, the dataset for this study resulted in a total of 30 recordings across a diverse group of participants consisting of men and women with ages ranging from the 23-78. The spontaneous speech and large variety of interview questions combined with dataset size may indicate some limitations for the result. During the data analysis it was found that pitch and HNR are significantly diverged for male and females [APPENDIX]. This has most likely impacted the results for all analyses conducted on the full dataset, by the reason that these features even out by the other gender and may create a normalised average even if it would be distinct if analysed separately. 
Given that the dataset for this research consisted of interviews capturing spontaneous speech, a broader range of vocal features might have contributed to the detection of the complex vocal patterns and given a more
nuanced understanding of the correlations for emotion recognition in the Swedish language. While the selected vocal markings chosen for this study gave some insight into addressing RQ1, expanding the set of features could have helped address RQ1 more comprehensively.


\medskip
\subsubsection{ALL RQ } 
- styrkor: replicerbar. kan testas mer kontr. + störrre dataset 

- hade varit fördelaktigt att tidssegmentera fler delar i rq1 

- intervju miljö 

- olika tid i intervjuer 

- kunde varit fokuserat på enbart två eller tre känslor 


\medskip

The interviews were designed to provoke one positive and one negative emotion, yet not directly oriented towards one of the three negatively oriented emotions. This structure was motivated by allowing the participants to talk freely about a subject they related to and felt comfortable to discuss. During the self-report after each interview, several participants raised confusion about how to interpret and rate surprise. This emotion showed most inconsistency throughout all research questions and was not aimed to be induced through the interviews. By reducing this study to only focus on one positive and one negative emotion, both for interview design and in data analysis, a more comprehensive analyse of only these emotions could be conducted to yield a narrower, yet deeper analysis of two focus emotions. On the other hand, patterns of vocal features for certain Hume-labelled emotions were benefited by enable comparison with all emotions used in the Swedish study. 

Hume AI was one of the models used and provided some advantages such as avoiding manual
labeling and being pre-trained, although the Hume AI emotion scores had to be normalized and
the emotions were filtered to use only the specific five emotions necessary for the comparisons
in this research, which may have had some limitations on the model’s capacity. Along with
working well for the research’s purpose, the model has some downsides. For example, there is
limited publicly available information about functions of the model, making it difficult to fully
assess possible limitations and biases. Despite these limitations, Hume AI contributed with valuable insights in answering RQ1.

Comparing our interview-based result with a larger and more controlled dataset with acted emotions could possibly have validated some observed patterns or strengthened the opposite whereas acted speech is expressed explicitly different from speech in real-world similar contexts. 

\subsection{RQ2 and RQ3 Methodological Considerations}
The methodological approach to address RQ2 combines analysis of transcribed text in emotion recognition using NLP Cloud in order to assess the emotional content of speech transcripts in relation to speech-based AI models.

In addressing RQ3, the approach was to compare self-reported emotions with the AI-generated labels from both speech and text-based models to analyze potential alignments.
This is a multi-modal approach with several methodological considerations, but also some methodological strengths.

The vocal recordings were transcribed and analyzed with the text-based emotion recognition tool NLP Cloud and self-assessments of emotions were collected after each interview, allowing a comparison between speech-based AI, text-based AI and self-assessment scores given by the participants in the interviews. 
The use of three different methods resulted in triangulation, which increased the flexibility and credibility in the findings. In addition to this, the usage of pre-trained AI models ensured consistent processing. However, some information loss was expected for the transcript text analyzed in NLP Cloud. When the model takes in what is said rather than how it is said, many important emotional cues such as intensity or pitch get lost. This could possibly have led to some emotions being misinterpreted or not catching the full complexity of the emotions expressed, based on only the text-based analysis.

While NLP Cloud contributed in addressing RQ2, some limitations in loss of prosodic information may have reduced the full emotional understanding.
The self-reported emotions introduced a valuable reference point for this research. Some agreement was found between the AI models and self-reported emotions, but some of the self-assessed scores may also have been slightly exaggerated. The emotional memories and personal interpretations of emotions by participants can have influenced the self-assessed emotion scores.
While the self-reported emotion scores have some limitations which likely contributed to some variability in the analyses, they helped valuably address RQ3.

The complexity of emotion detection across different modalities is highlighted by the AI models being able to capture some emotions in a quite robust way while struggling more with others. The few emotional categories used for this research may have limited the emotion recognition, where an implementation of more emotions and features possibly could have captured the emotions in a better way.

All methods used in answering RQ2 and RQ3 provided valuable information and findings, however the research could have benefited from an expansion of the emotion categories to help identify emotions in a more accurate and complex way.

\subsection{Summary of Methodological Considerations}
To address all research questions, this study utilized a multi-method approach combining speech-based and text-based emotion recognition with self-reported emotion scores of the participants from the interviews. 

Although all methods contributed with important findings and significant insights into emotional expression in Swedish speech, a number of limitations emerged.

While the triangulation of speech, text and self-assessment scores contributed to the strength and credibility of the findings, size of dataset, model transparency and other limitations such as variabilities and inconsistency in having spontaneous interviews may have impacted the effectiveness of the findings. Although highlighting some areas for improvement for future studies, the methods chosen for this research overall contributed to answering the research questions in a comprehensive manner.
