\chapter{Discussion}
\label{sec:discussion}

\section{Result Discussion RQ1}
For the first research question, this thesis investigated how AI models for speech recognition compare to existing research on vocal markers. More specifically, the goal is to assess whether the AI models align with the findings of the Swedish research on vocal markers done by Ekberg \autocite{Ekberg2023}.
\subsection{Interpretation of Results}
\subsubsection{Vocal Features and Hume AI Emotion Scores}
Correlations between different vocal markers (pitch, intensity, HNR, jitter and simmer) and Hume AI emotion labels, were visualized on a heatmap using Pearson correlation coefficients. Overall, low to very low correlations were found here, with only a few instances of what could be considered moderate correlations.  
Where values closer to 1.0 for positive linear relationships or -1.0 for negative linear relationships would suggest a strong correlation, the heatmap showing the correlation between the vocal features and AI emotion scores from Hume AI primarily showed low numbers ranging from -0.44 as the lowest, to 0.34 as the highest. Some numbers appeared to have an extremely low correlation. For example, the correlation between sadness and shimmer was as low as 0.00, with other correlations hovering around being 0.01-0.04.
This suggests a weak linear relationship between the vocal features mean pitch, mean intensity, mean HNR, jitter and simmer and the emotion labels from Hume AI, indicating that the AI model may not fully have captured the complex details from the vocal markers as well as the findings from Ekberg \autocite{Ekberg2023}.

Despite this, some of the results align with the findings of Ekberg. 
For example, in this analysis mean pitch with anger shows the value r = 0.26 while mean pitch with sadness gave the value r = -0.37. Ekberg’s research showed anger to have elevated pitch and while sadness had lower pitch than anger and happiness, so both studies show anger to be associated with elevated pitch and sadness to be associated with lower pitch.
Looking at intensity, this analysis showed the value of joy to be r = 0.34, sadness r = -0.26, and surprise r = -0.39. In Ekberg’s study, happiness (joy) showed higher intensity while sadness and surprise showed patterns of lower intensity. Although these emotions show matching patterns, other emotions show mismatches. Ekberg’s study reported higher intensity for anger and fear, while the heatmap analysis in this study presents fear to have a minimal correlation (r = 0.01), while anger has a negative correlation (r = -0.18) with intensity, suggesting that intensity decreasing as anger increases.

The findings for HNR only partially matched with the findings of Ekberg’s study which reported that fear and happiness were linked with higher HNR while sadness is associated to lower HNR. 
In this study sadness showed a moderate negative correlation with HNR (r = -0.44), which is aligns with the Ekberg study. However, correlations for fear (r = 0.08) and joy (r = 0.06) were very weak, differing from the associations observed by Ekberg.
Jitter showed no moderate or strong correlations with any emotions in this study, and shimmer proved to be slightly positive for joy (r = 0.19) and moderately negative for anger (r = -0.33) but not for surprise which was the only emotion in Ekberg’s study which had higher shimmer.

Although some Pearson correlations presented in the heatmap showed consisting pattern with the Ekberg research and others diverged, it is important to note that there are methodological differences between this analysis and Ekberg’s research. Ekberg employed a different statistical approach using both simple and multiple logistic regression models to predict the emotions from speech. While Pearson correlation which was used for this analysis is useful for detecting linear associations, it might not have captured the complex non-linear interactions that the logistic regression models are able to capture.
The weak correlations found in the heatmap showing the correlation between the vocal features and AI emotion scores from Hume AI might also suggest that the vocal features used in this analysis were insufficient in predicting the emotions in spontaneous speech. Where the research done by Ekberg uses an acted dataset with repeated sentences, emotional expression in natural speech (e.g. interviews, which were used in this analysis) tends to be more subtle.

\subsubsection{Praat-Based Emotion Scores}
As for the second heatmap presented in the result, we see the correlations between vocal features and the emotion scores from the custom Praat-based categorization function. While the Pearson correlation values in this heatmap are seemingly stronger than with the Hume AI labels, these results may be misleading as they reflect an over-reliance on pitch and HNR, rather than authentic emotional differences.
For instance, examining the highest correlations, pitch correlated strongly with fear (r = 0.93) and surprise (r = -0.90). HNR also demonstrated high correlations with fear (r = 0.92) and surprise (r = -0.90). 

While the correlations here suggest a strong relationship between the emotional labels and the vocal features, this specific pattern indicates methodological limitations. The Praat-based categorization function appears to prioritize a narrow range of features (notably pitch and HNR), leading to inflated correlations that do not necessarily capture the complexity of emotional expression.

The failure of accurately categorizing vocal emotions is likely due to the spontaneous nature of the interviews which unlike acted datasets where emotions are exaggerated, result in subtle emotional expression. The limited number of vocal features may also be a contributing factor as to why full emotional complexity was not captured, together with variability in recordings which potentially diluted the emotional markers across time.
Contrary to theoretical expectations, the categorization failed to distinguish between the emotional states in a meaningful way as indicated by the lack of clear emotional differentiations. These findings suggest that although some acoustic features were captured effectively, the Praat-based function did not categorize the emotions accurately, potentially due to the speech being in a spontaneous interview format instead of, for example, an acted dataset. 

\subsubsection{Custom Vocal Emotion Categorization Method}
To assess the effectiveness of the custom emotion categorization function developed for this study, the outputs were compared to the Hume AI generated emotion scores.

Despite individual vocal features having high correlations, the Praat-based emotion categorization function reveals significant limitations. With an average score of 0.2, the Praat based function presented minimal variability, suggesting that the function was unable to differentiate emotional expressions within the spontaneous speech obtained from the interviews.

The Hume model demonstrated wider variations for the emotion scores, reflecting on a more nuanced detection of the different emotions. 
Several factors likely contributed to this outcome as these results highlight the challenge of categorization in spontaneous speech. In real world emotional expressions, as opposed to acted ones, emotional signals are often more subtle and dependent on context. 

The limited set of acoustic features may as well have excluded some important clues, further restricting the sensitivity of the function. All factors point to spontaneous speech potentially being too complex for this function.

\subsubsection{ANOVA Vocal Features}
To further investigate whether fundamental vocal features used in the previous analyses varied systematically across the five different AI-labeled emotions, an analysis of variance (ANOVA) was conducted. None of the results revealed any significant statistical differences for pitch, intensity, HNR, jitter or shimmer. This was further confirmed by Tukey HSD tests, suggesting that within the context of spontaneous speech obtained from the interviews, emotional states may not reliably be differentiated by average values of the core vocal features.

Clear differences in acoustic features between emotions were reported by Ekberg \autocite{Ekberg2023}, which contrasts with these findings, possibly due to the controlled nature of Ekberg’s acted dataset.
The lack of variance for the present study likely reflects the subtle nature of emotional expressions in an interview format, being a natural conversational form of expression with some emotions possibly being interwoven with context and which may also have more variations from time to time.

\subsubsection{Correlation Between Vocal Features and Hume AI Emotion Scores}
Following the limitations identified with the rule-based emotion categorization, the analysis shifted to examining direct correlations between raw acoustic features and Hume AI’s emotion scores. 
With a composite correlation analysis visualized, showing the two acoustic features pitch and intensity and the correlation with Pearson correlation coefficients, generally weak correlations were found across all emotions. 

However, some patterns aligned with established findings of Ekberg \autocite{Ekberg2023}, where intensity showed a negative correlation with fear, sadness and surprise which bears some resemblance with the results of Ekberg’s study. Although pitch presented minor positive correlations with anger and fear, and happiness, these results alone suggest that average pitch and intensity alone are insufficient in capturing the complex nature of emotional expression made in spontaneous speech, as it remains too subtle. Although some expected relationships have been observed, emotions fluctuate dynamically within a clip, contributing to the difficulty of detecting consistent patterns through static averages of single features.

\subsubsection{Observations from individual interviews}
Analyses of individual interviews revealed results can give more clarity by analyzing a voice recording in its entirety combined with looking at peaks in specific moments. Where static measures often result in an average of emotions when analyzing an entire voice recording, segment analysis done from time to time shows more detailed emotional shifts. Although correlations are not entirely consistent throughout the recordings, the positive interview revealed increasing intensity corresponding with higher joy, whereas one of the negative voice recordings revealed peaks in pitch aligned with elevated anger scores. 
There are several possible factors that can account for this result, one being that the emotional expression in natural conversations such as the interview format for this study is context driven with emotions fluctuating throughout the entire conversation. Static measurements take an average of all the peaks and fluctuations, likely resulting in a somewhat neutral end result, where datasets with acted recordings might not differ as much in one recording.
These findings highlight the importance of analyzing emotions on a segment level rather than relying exclusively on an average for one whole clip. Looking at one clip at its entirety for a natural conversational clip gives the opportunity to inspect all emotional peaks and distinguish between different emotional states throughout the recording without emotions being averaged and neutralized.
Furthermore some variability across individual speakers was observed demonstrating different vocal patterns suggesting that some personal vocal traits can impact the level of detectability of emotions. Some of these patterns could be due to factors such as gender or speaking style.

\subsection{Limitations and Explanations}
There are several limitations which should be taken into consideration when interpreting the results presented for the first research question.

Firstly, the dataset conducted and used for this study consist of spontaneous conversational semi-structured interviews. This likely has resulted in more subtle emotions than an acted dataset would contain with weaker emotional expressions reducing the detectability of different vocal markers. 
This study has also been restricted to a small set of acoustic features for the analysis. There are possibilities other acoustic features would have contributed with relevant aspects for the analyses if included. This study does not involve all vocal features that are included in the research by Ekberg \autocite{Ekberg2023} which potentially could have shown relevant information and possible alignments, however, due to the timeframe and scope of this study only five vocal markers were chosen.

Another important limitation to consider is the static averages of the vocal features. Despite effort where some clips have been visualized in its entirety, showing spikes in one specific emotion combined with one specific vocal feature from time to time, most clips involved in the process of answering the first research question are the result of an average per clip. This may have obscured some dynamic fluctuations of all emotions over time.

A further constraint worth noting is that the Hume AI emotion scores are not perfect estimates of true emotional states themselves and should not be considered ground truth in this study, only a way of comparison and finding potential similarities. Additionally, although Hume showed varied and appropriate emotion outputs, it is important to note that the Hume AI emotion outputs are based on soft scoring. This often produces mixed emotions rather than single emotional states, making comparisons more difficult.

Finally, the usage of the Swedish vocal data collected from semi-structured interviews. These interviews consist of spontaneous and conversational speech which likely do not involve as strong emotional expressions as acted datasets. Conversational speech tends to be more subtle and may have a reduced level of clear vocal markers. 

Worth noting is that the interviews conducted for this study were context-dependent and influenced by topics selected by the participants of the interviews themselves, introducing potential additional variability from interview to interview. Contrary, the Swedish research by Ekberg consisted of 14 repeated sentences.

\subsection{Conclusion for RQ1}
The result for the first research question has investigated if AI-based emotion recognition models align with existing research on vocal markers with a focus on the Swedish language.
Exploring vocal markers correlation with Hume AI emotion labels as well as the correlation between vocal markers and Praat, the overall result demonstrated limited strength.
Overall the results showed weak or moderate correlations, although some relevant patterns aligning with the existing research on Swedish vocal markers \autocite{Ekberg2023} was found.
While the values from the Hume correlations appeared moderate at best, the Praat correlations overall showed weak correlations except some misleading numbers suggesting an over-reliance on the vocal markers pitch and HNR.

The segment level analysis gave important insight into the fluctuations in the emotions throughout entire clips compared to an average value of the different emotions.

The result indicates that while there are certain vocal features that remain relevant as indicators of emotional states, spontaneous speech presents challenges in emotion recognition. In comparison to acted datasets, emotions are more subtle with more variety and contextual dependence in spontaneous speech.
Though there are challenges with spontaneous speech, the result suggest that AI-based emotion recognition systems such as Hume AI showed promise, demonstrating some flexibility and context-awareness.

Future work could benefit from incorporating a wider range of vocal features, emotions and a more dynamic approaches to capture the complexity of emotional expression.

\section{Result Discussion RQ2 and RQ3}
For the second research question in this thesis the aim was to investigate whether we could understand the emotions from textual content of the speech, with the same data as in RQ1. This was achieved by transcribing the vocal recordings and analyzing them using NLP Cloud’s emotion recognition to detect emotions in the textual content of the speech.

For the third and final research question, the objective was to assess how the AI generated emotion labels obtained through the speech-based and text-based emotion recognition would compare to the self-reported emotions provided by the interviewees.

\subsection{Interpretation of Results}
\subsection{RQ2: Speech-based AI vs Text-based AI}
In the comparison between the speech-based emotion recognition model, Hume AI, and the text-based emotion recognition model NLP Cloud, the system overall seemed to show some levels of agreement for certain emotions. Using both descriptive statistics and visual analyses to calculate the differences, an overall comparison of both ai systems showed that the mean emotion scores differ across the two models.
The average difference in the emotion scores showed values indicating that Hume AI obtained higher scores for the emotions anger and fear, while NLP Cloud proved to show higher scores for joy, sadness and surprise. Despite these findings, the score for sadness and surprise were sufficiently low, suggesting that the models were substantially aligned on specifically those emotions.
Joy being highly scored by NLP Cloud indicates that joy may not have been as easily identified in speech-based emotion analysis, while the textual context may have conveyed a more positive tone from the text than appeared in the voice. In contrast, anger and fear appeared to have been more effectively captured by the speech-based emotion detection, possibly suggesting that someone might sound angry or fearful even though they may not be experiencing these emotions in the moment.
Based on the Pearson correlation analysis showing the association between the text-based and speech-based emotion recognitions, the strongest alignments were shown for Joy (r = 0.521) and anger (r = 0.468). Joy and anger also showed statistically significant p-values, where joy had a p-value of 0.0069, and anger had a p-value of 0.0022. No further strong correlations or statistically significant p-values were found in the other emotions.
Several factors may account for this result. For example, joy and anger are distinct emotions, while sadness, fear and surprise may likely involve more subtle cues and contextual factors. Being more complex to detect may have contributed to the lower consistency across the two models for these specific emotions.

For the full dataset, paired t-tests showed no significant differences for the mean score of the emotions across the dataset for all emotions except fear. Although the correlation between Hume AI and NLP Cloud showed non-significant scores for fear, the t-test indicated that while the systems do not align on detecting patterns for fear, Hume AI consistently rates the fear higher than NLP Cloud. Possible explanations for this result may reflect the differences in how emotions are conveyed and detected in the different models, whereas Hume AI possibly could have captured the more subtle vocal indicators that might not have been as easily expressed or detected in text.

Examining the t-tests for the positive oriented interviews in comparison to the negative oriented interviews, notable findings emerged.
For the positive interviews, significant differences between Hume AI and NLP Cloud were found for all emotions with the exception of surprise, where Hume consistently detected higher levels of sadness and fear and NLP on the other hand overestimated joy and anger in comparison to Hume.
This may be explained by the complexity of emotions and emotional expression. In the positive interviews, the participants discussed joyful topics, and while this may have been detected for the text-based emotion recognition, the vocal tone could reveal more subtle cues in the tone, rhythm and pitch. For a positive interview, the participant may have a lower and more neutral tone and pitch than an actor acting out happiness, which could be one explanation for this result.
For the negative interviews, significant differences were only identified for joy and sadness, where Hume rated joy with a higher score, and NLP rated sadness higher. This indicates a better alignment for the different models for the analyses made for the negatively oriented interviews.

Possible explanations for these results are that people participating in the interviews may have used overly positive language out of politeness, even if the content of the words may have been negative.
\subsubsection{Sentiment-Based Analysis RQ2}
In comparing Hume AI and NLP Cloud, the sentiment-based analysis presented a distinct pattern in how the different AI models interpreted the positively oriented interviews versus the negatively oriented interviews, where some emotions were consistently rated higher than others.

NLP Cloud showed patterns of consistently rating joy higher than Hume AI, while Hume rated higher for negative emotions such as anger, sadness and fear in the positive interviews.

These results indicate that the subtle features such as pitch, loudness and more may have been interpreted by Hume as negative emotions even in positive conversations.
Surprise remained a challenging emotion to detect, and NLP Cloud showed higher values of anger and sadness in the negative interviews, likely due to being able to better capture the negative context of the interviews through the text-based analysis.
Hume rated joy unexpectedly high in the negative interviews, where a possible reason could be nervous laughter or other emotions that could have been misclassified.
Overall, the results underscores that the two AI models differ in the job of emotion detection, possibly due to the vocal recordings involving subtly expressed emotions or possible irony or laughter that could have been incorrectly categorized as joy. 

\subsection{RQ3: AI vs Self-Assessed Emotions}
For the third and final research question the alignment with the speech-based emotion labels from Hume AI and the text-based emotion labels from NLP Cloud in combination with self-assessed emotion scores were examined. Insightful findings revealed some levels of alignment dependent on both the model and emotion.
With an analysis showing an average of the emotion scores across the entire dataset of interviews, joy emerged as the emotion with the highest average scores. Fear and surprise showed the lowest scores out of the emotions. A visualization of the average emotion scores across all three channels shows NLP Cloud overestimating joy excessively, while Hume on the other hand overestimated anger to a certain degree.
The rest of the emotions are relatively close in scores across the models and self-assessment scores, where sadness, fear and surprise were rated low for all channels.
The low scores overall for sadness, fear and surprise could be explained by the spontaneous interview format in a calm setting, which may not encourage expressively conveying these emotions.

For the case of joy having a substantially higher score for NLP Cloud, the model may have interpreted language as joyful even though the tone was more neutral, also possibly missing out on cues such as irony.
Hume estimated anger higher than NLP Cloud and the interviewees themselves, which may be due to misinterpreted signs of anger for example from pitch and intensity.

\subsubsection{Hume AI and NLP Cloud vs Self-Reported Emotions}
The correlations between Hume AI and the self-reported emotions indicated very weak to modest correlations for all emotions, where only anger showed a modest statistic significant correlation with a Pearson value of r = 0.359 and p-value = 0.043. This may be due to the nature of anger which often produces a distinct vocal change typically involving increased loudness and change of pitch, while emotions like fear and surprise may often be expressed with more subtle vocal expressions which may not have been captured as successfully. Once again, the calm setting in an interview environment might also have affected the results, further muting emotional expressions. This suggest that although Hume moderately detect some emotions based on voice, a multimodal approach with further analysis might be necessary for more complex emotion recognition.

The results for NLP Cloud presented statistically significant correlations for all emotions except surprise. This indicate a high degree of alignment between the self-reported emotions and the text-based emotion detection NLP Cloud, where the strongest correlations were presented in joy (r = 0.863, p = 0.0000), anger (r = 0.739, p = 0.0000), sadness (r = 0.710, p = 0.0001), and lastly fear (r = 0.669, p = 0.0003).
This strongly indicates the effectiveness of NLP Cloud in capturing emotional content through text in combination with alignment with self-reported emotions gathered from the interviews. 

Surprise being the only emotion not to show a strong correlation, emphasizes a consistent challenge observed over both models used in the study. During the self-evaluation segment of the interviews, multiple participants expressed certain confusion regarding the assessment of the emotion surprise. A large part of the interviews consisted of describing past emotional experiences which may have reduced the intensity of surprise. Typically, surprise is expressed as an immediate reaction to unexpected events and its unlikely that the interviewees are able to genuinely experience the same surprise felt in the original moment of the memory. This provides a possible explanation for why both AI models overall detected low levels of surprise, while an acted dataset could present higher correlations for this emotion.
The statistical analysis made evident that both Hume AI and NLP Cloud showed partial alignment with the self-assessed values for some emotions. Hume AI showed a larger deviation for anger and surprise, whereas NLP Cloud deviated more for fear and joy. This suggest that the effectiveness of the AI models in emotion detection is not consistent across the emotions, although the challenging nature of self-assessing emotions, especially fear and surprise, retrospectively possibly complicates this process, effecting the results as well.


\subsubsection{Sentiment-Based Analysis RQ3}
In the sentiment-based analysis comparing Hume AI, NLP Cloud, and self-reported emotions insight was gained into how the AI models align with the personal perceptions of emotions.
In the positively oriented interviews, NLP showed higher rating of joy compared to the self-assessed scores, while Hume AI consistently rated joy lower than the self-assessed scores in combination with detecting higher levels of all negative emotions (anger, sadness and fear).

These results suggest that subtle vocal markers were captured by Hume that may not have matched the content.
For surprise, NLP Cloud closely matched the scores of the self-assessment whereas Hume AI detected lower levels. This may reflect all challenges previously mentioned in regard to the emotion surprise.

In the negatively oriented interviews, both fear and surprise were relatively evenly rated across all three sources, with the self-assessed being the highest rated in both emotions.

The ratings for anger were high for all sources as well and fairly evenly matched between Hume and the self-assessed scores, while NLP Cloud rated anger higher. The higher rating by NLP Cloud was likely due to the context of negative wording in the negative interviews.
Hume AI rated sadness low, while NLP Cloud was relatively close in score compared to the self-scores, suggesting the text-based model might have been better at capturing sad emotions from the vocal recordings. Hume may not have picked up the cues for sadness in the same capacity, likely due to the low expressions of sadness during the interviews. 
Further analysis showed Hume AI rating joy higher than both NLP Cloud and the self-reported emotions in the negative interviews, likely due to misclassifications of certain vocal cues that may have been subtle or complex, for example irony which could be difficult for a speech-based AI to recognize. 
Overall, the text-based AI NLP Cloud seems to align closer with the self-assessed scores rated by the participants of the interviews, possibly capturing the context for each interview more effectively. This underscores the limitations of relying exclusively on either speech-based or text-based emotion recognition.

\subsection{Limitations and Explanations}
This study has presented several important insights in emotion detection using AI models, although there are several limitations that should be noted.
The spontaneous nature of the interviews remains a limitation through RQ2 and RQ3, as these vocal recordings may have given more subtle and muted emotional expressions compared to an acted dataset would. The calm setting of the interviews may also be an explanation to why fear and surprise especially was not detected to a high degree.

The self-reported emotions unavoidably involve subjective biases, which possibly could have resulted in some variety across interviews. 
The dataset size and the limited emotions remains a limitation, where a larger dataset and more emotions possibly could lead to broader findings and correlations.

\subsection{Conclusion for RQ2 and RQ3}
In answering RQ2 and RQ3, this study explored effectiveness of speech-based emotion recognition, Hume AI, text-based emotion recognition, NLP Cloud. These were later examined for potential alignments with self-assessed scores for emotions.
For RQ2, partial agreements were found between the two AI models for some particular emotions such as joy and anger, though notable disagreements were present for fear and surprise. 

This brought attention to the challenges of detecting the emotional cues for more complex emotions, where they might have more subtle cues for detection.
In comparisons between the models, certain differences in how each model captured emotions were found.

For RQ3, the comparison of the models with the self-assessed emotion scores indicated a stronger consistency for NLP Cloud than it did for Hume AI.
Overall, surprise was presented as an emotion consistently challenging to detect across the models, possibly due to the complexity of the emotion combined with the nature of the interviews where this emotion may have been expressed the least.

The results for this study suggest that a multimodal approach integrating multiple sources could enhance the precision and reliability of the detection systems for emotions, as relying on only one type of analysis may not be enough for the complexity of emotions.

\section{Method Discussion}
\subsection{RQ1 Methodological Considerations}
To answer RQ1, the methodological approach involved analysis of emotional expression for vocal markers in Swedish speech in comparison to AI based emotion recognition models. The idea was to analyze emotions in a clip in its entirety and find correlations, which had some differing results, but it proved to be a notable strength to execute the analysis on a segment level to capture emotional fluctuations in a more dynamic way. While this offers another another perspective, this approach introduced challenges of its own in having some inconsistent emotions not aligning completely across the segments. Therefore the methodological approach was partially fulfilled for answering RQ1 by identifying some emotional fluctuations, while also revealing challenges in both the analyses for segment-leveled clips and full clips.

One of the studies chosen to compare the results with, being the existing Swedish emotion research by Ekberg (Ekberg et al., 2023) proved some similarities and patterns which provided valuable information to this study.  However, the research used pre-defined sentences, repeated by actors, which may have given a more consistent result than the dataset used in this research which consisted of interviews capturing spontaneous speech. With 16 participants, the dataset resulted in a total of 32 recordings across a diverse group of participants consisting of men and women with ages ranging from the 23-78. The spontaneous speech and large variety of interview questions combined with dataset size may indicate some limitations for the result. While RQ1 was addressed, a larger and more controlled dataset with acted emotions along with repeated sentences, could possibly have validated some observed patterns, ensuring more consistent emotions throughout the recordings.

Hume AI was one of the models used and provided some advantages such as avoiding manual labeling and being pre-trained, although the Hume AI emotion scores had to be normalized and the emotions were filtered to use only the specific five emotions necessary for the comparisons in this research, which may have had some limitations on the model’s capacity. Along with working well for the research’s purpose, the model has some downsides. For example, there is limited publicly available information about functions of the model, making it difficult to fully assess possible limitations and biases.

Despite these limitations, Hume AI contributed with valuable insights in answering RQ1. 

A set of basic vocal features which consisted of pitch, intensity, harmonic-to-noise ratio (HNR), jitter, shimmer, was extracted through Praat. These features are well established indicators of emotional expression but proved to be somewhat of a limitation which possibly could have been avoided by incorporating additional vocal features. Given that the dataset for this research consisted of interviews capturing spontaneous speech, a broader range of vocal features might have contributed to the detection of the complex vocal patterns and given a more nuanced understanding of the correlations for emotion recognition in the Swedish language.

While the selected vocal markings chosen for this study gave some insight into addressing RQ1, expanding the set of features could have helped address RQ1 more comprehensively. 

\subsection{RQ2 and RQ3 Methodological Considerations}
The methodological approach to address RQ2 combines analysis of transcribed text in emotion recognition using NLP Cloud in order to assess the emotional content of speech transcripts in relation to speech-based AI models.

In addressing RQ3, the approach was to compare self-reported emotions with the AI-generated labels from both speech and text-based models to analyze potential alignments.
This is a multi-modal approach with several methodological considerations, but also some methodological strengths.

The vocal recordings were transcribed and analyzed with the text-based emotion recognition tool NLP Cloud and self-assessments of emotions were collected after each interview, allowing a comparison between speech-based AI, text-based AI and self-assessment scores given by the participants in the interviews. 
The use of three different methods resulted in triangulation, which increased the flexibility and credibility in the findings. In addition to this, the usage of pre-trained AI models ensured consistent processing. However, some information loss was expected for the transcript text analyzed in NLP Cloud. When the model takes in what is said rather than how it is said, many important emotional cues such as intensity or pitch get lost. This could possibly have led to some emotions being misinterpreted or not catching the full complexity of the emotions expressed, based on only the text-based analysis.

While NLP Cloud contributed in addressing RQ2, some limitations in loss of prosodic information may have reduced the full emotional understanding.
The self-reported emotions introduced a valuable reference point for this research. Some agreement was found between the AI models and self-reported emotions, but some of the self-assessed scores may also have been slightly exaggerated. The emotional memories and personal interpretations of emotions by participants can have influenced the self-assessed emotion scores.
While the self-reported emotion scores have some limitations which likely contributed to some variability in the analyses, they helped valuably address RQ3.

The complexity of emotion detection across different modalities is highlighted by the AI models being able to capture some emotions in a quite robust way while struggling more with others. The few emotional categories used for this research may have limited the emotion recognition, where an implementation of more emotions and features possibly could have captured the emotions in a better way.

All methods used in answering RQ2 and RQ3 provided valuable information and findings, however the research could have benefited from an expansion of the emotion categories to help identify emotions in a more accurate and complex way.

\subsection{Summary of Methodological Considerations}
To address all research questions, this study utilized a multi-method approach combining speech-based and text-based emotion recognition with self-reported emotion scores of the participants from the interviews. 

Although all methods contributed with important findings and significant insights into emotional expression in Swedish speech, a number of limitations emerged.

While the triangulation of speech, text and self-assessment scores contributed to the strength and credibility of the findings, size of dataset, model transparency and other limitations such as variabilities and inconsistency in having spontaneous interviews may have impacted the effectiveness of the findings. Although highlighting some areas for improvement for future studies, the methods chosen for this research overall contributed to answering the research questions in a comprehensive manner.